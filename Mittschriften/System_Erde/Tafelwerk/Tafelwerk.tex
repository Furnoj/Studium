\documentclass[a4,12pt]{scrartcl}

\usepackage{amsmath}
\usepackage{graphicx}


\begin{document}

\title{Systhem Erde Tafelwerk} \label{sec:Systhem Erde Tafelwerk}
\maketitle




\section{Fachbegriffe und Definitionen} \label{Fachbegriffe und Definitionen}

\begin{enumerate}
\item Minerale: Festkörper der eine Gitterstruktur besitzt, natürlich vorkommt und eine definierte Zusammensetztung aufweist
\item Koordinationszahl (CN): Gibt an wie viele Ionen in unmittelbarer Umgebung sind 
\item Interstellare Gaswolke: diffuse Nebel in dene sich neue Sterne bilden
\item protosolarer Nebel/ Urnebel: Interstellare Gaswolke der Sonne
\item Akkretion Bildung von Massereichen Körpern durch zusammenschluss leichterer Materie
\item Revolution: Umlauf eines Objekts z.B um die Sonne
\item Rotation: Drehung eines Objektes um sich selbst
\item Exzentrizität: Abweichung von der Kreisbahn
\item Obliquität: Schiefe der Erdachse
\item Präzession: Orientierung der Kreiselachse
\item Meteoroi Kleinkörper des Sonnensystems, der als Komet oder Asteroid die Sonne umkreist
\item Meteor: Leuchterscheinung in der Atmosphäre
\item Meteorit: Objekt, das auf der Erde auftrifft. In der Regel entweder metallisch (Eisenmeterorit) oder aus Stein (Steinmeteorit)
\item Chondrite: Meteoriten mitSilikatkügelchen, eingebettet in eine Feinkörnige Grundmasse
\item Achondrite
\item Chondren: kuglige Minerale
Fluchtgeschwindigkeit: Geschwindigkeit, die ein Körper braucht um der Erde zu "entkommen"
\item Gezeiten: Gezeiten beschreiben die Verformung eines Körpers, 
die durch die Schwereanziehung eines Nachbarkörpers erzeugt 
wird.
Verwerfungen:
\begin{enumerate}
\item Blattverschiebung: (seitwerz)
\item Abwerfung: (relativ abwerts)
\item Aufwerfung: relativ aufwertz
\end{enumerate}
Liquefaktion: Wasser aus tieferer Umgebeung gelangt durch die Auflockerung des Erdbebens in obere Schichten und lockert den Boden auf\\
\item adiabatischer Effekt: überführung eines Systhem ohne Wärmeenergieänderung.
\item Besaltisch: niedrige Konzentration vernetzter $SiO_2$
\item Rhyolitisch hohe Konzentration vernetzter $SiO_2$
\item \begin{tabular}{ll}
Halbspreizungsrate: & langsam (Ridge): <3cm/a \\
 & schnell ()Rise >3cm/a
\end{tabular}
\item Ophiolithe: Ozeanische Kruste, die tektonisch auf kontinentale Kruste übergeschoben wurde (Obduktion)
\item Plumes: Aufsteigende Materialströhme im Mantel, welche an der Oberfläche pilzförmige strukturen bildet
\item Alkalien/$SiO_2$ kann ein Maß für die Masse an Schmelze sein:\\
\begin{tabular}{ll}
wenig Alkalien & viel Schmelze\\
viele Alkalien & wenig Schmelze\\
\end{tabular} 
\item Tiefseegraben: Rille die sich zwischen der abtauchenden und drüberschiebenden Platte ergiebt
\item Magmatischer Bogen: Bogen an dem vulkanische Aktivität mit basaltischen Schmelzen herrscht
\item Anwachsteil: Fläche zwischen Magmatischem Bogen und Tiefseegraben
\item Forearc: 
\item Backarc:
\item Delamination
\item Orogenese: Gebirgsbildung
\item Orogen: Gebirge
\item Fazien: Temperatur-Druck-Bereich
\item Faltenüberschiebungsgürtel : bei Orogenbildung entstehenden Umfaltungsschichten.
\item Molasse: Sedimentablagerung hinter Orogenen
\item Sedimente: lockermaterialien unterschiedlichster Herkunft, z.B Erosion, Vulkanismus, Organismen, chemische Ausfällung
\item Sedimentgesteine: verfestigtes Sediment
\item 
\end{enumerate}

\section{Formeln} \label{Formeln}

\textbf{Ionische Bindungslänge:}\\
$L=r_c + r_a$\\
\\
\textbf{Fluchtgeschwindigkeit:}\\
$v_{Flucht} = \sqrt{\frac{2MG}{r}}$
\\
\textbf{Geschwindigkeit seismischer Wellen:}\\
$v = \sqrt{\frac{\text{Rückstellspannung}}{\text{Masse}}}$\\
$v_p = \sqrt{\frac{k + \frac{3}{4} \mu}{\varrho}}$\\
$v_s = \sqrt{\frac{\mu}{\varrho}}$\\
\textbf{Druck}\\
$P = h \cdot \varrho \cdot \overrightarrow{g}$\\
$\overrightarrow{g} = g_0 \cdot \frac{r}{r_\odot}$\\
\textbf{Aufttrieb}\\
$h_{Freibord} = h_{gesamt} \cdot (1 - \frac{\varrho_{Objekt}}{\varrho_{Fluid}})$\\
\textbf{Wärmeleitung}\\
$\frac{q}{A} = - \frac{\delta T}{\delta Z} \cdot K$\\
\begin{tabular}{ll}
q & 	Wärmeleistung<\\
A & Fläche\\
T & Temperatur\\
Z & Dicke\\
K & Wärmeleitfähigkeit\\
\end{tabular}\\
\textbf{Momenten-Magnetudenskala}\\
$M_W = \frac{2}{3} log_{10}(M_0)-10,7$\\

Seismisches Moment ($M_0$) = Scherfestigkeit (durch deformation gespeicherte Energie) $\cdot$ Ausdehnung der Rissfläche $\cdot$ Versatz \\
\textbf{Temperaturleitfähigkeit $\kappa$}\\
$\kappa = \frac{K}{\varrho \cdot C}$\\
$t = \frac{z^2}{\kappa}$\\
\begin{tabular}{ll}
K: & Wärmeleitfähigkeit\\
C: & Wärmekapazität\\
z: & Dicke
\end{tabular}
\end{document}