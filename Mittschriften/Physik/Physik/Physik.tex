\documentclass[a4paper,12pt]{scrartcl}

\usepackage{amsmath}
\usepackage{pgfplots}
\pgfplotsset{width=10cm,compat=1.9}

\begin{document}

\title{Physik}
\maketitle

\section{Methoden der Physik} \label{sec:Methoden der Physik}

\begin{itemize}
\item Aufbau eines Experimentes
\item Beschreibung
\item 3a) Formulierung des Ergebnisses in mathematische Form und Theorie
\item 4 Physikalische Bedeutung
\item 3b) Überprüfung der Theorie (weil vorherige falsch)
\item 3c) Erweiterung der Theorie (weil vorherige zu ungenau)

\subsection{Fehler} \label{sec:Fehler}
\subsubsection{Fehleranalyse} \label{sec:Fehleranalyse}

\item arithmetische Mittel zum Aussieben grober Messfehler
\item alternativ: Median (Robust gegen eingerechnete ausreißer)
\item Standartabweichung der Einzelmessungen zum feineren Aussieben der Messfehler
\item Standartabweichung des arithmetischen Mittels zur Überprüfund der Genauigkeit

\subsubsection{Fehlerarten} \label{sec:Fehlerarten}

\item statistische Fehler: Zufall/Reaktionszeit
\item systematische Fehler: Kallibration/Logikfehler

\subsubsection{Fehlerreduzierung} \label{sec:Fehlerreduzierung}

\item länger/öfter messen
\item besser messen (Lichtschranke)
\item systematische Fehler klein halten
\end{itemize}
\subsubsection{Fehlerfortpflanzung} \label{sec:Fehlerfortphlanzung}
Stellt man $f(x), x$ sei die fehlerhafte Variable, graphisch da, ist die Abweichung $\sigma = f(x \pm \sigma)$

% mind. 2 Vorlesungen fehlen


\section{Arbeit, Energie und Leistung}









\end{document}