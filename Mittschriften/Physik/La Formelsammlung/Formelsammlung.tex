



\documentclass[a4paper,12pt]{scrartcl}

\usepackage{amsmath}
\usepackage{graphicx}

\begin{document}

\title{Tafelwerk Physik}
\maketitle
\section{Formeln}

Median: 
\begin{center}
7,6; 0,843; $\boxed{0,98}$; 1,26; 1,46
\end{center} 

Schätzung des Mittelwertes: \\
\begin{align*}
\bar{x} = \dfrac{1}{n} \cdot \sum\limits_{i=1}^n (x_i)
\end{align*} 

Schätzung der Standartabweichung der Einzelwerte: \\
\begin{align*}
\Delta x = \sigma x = \sqrt{ \cfrac{\sum \limits_{i=1}^n (x_i - ]\bar{x})^2}{n-1}}
\end{align*}

Standartabweichung des Mittelwertes: \\
\begin{align*}
\Delta x = \sigma x = \sqrt{ \cfrac{\sum \limits_{i=1}^n (x_i - \bar{x})^2}{n \cdot (n-1)}}
\end{align*}

1. Raketengleichung\\
$m \cdot \overrightarrow{a} = v_t \cdot m(t)+\overrightarrow{g} \cdot m$

\begin{tabular}{ll}
$v_t$ & Austrittsgeschwindigkeit\\
k & k\\
\end{tabular}

Fehlerfortpflanzung:
\Large
\begin{multline*}
y = f(x_1;x_2;x_n) \\
\sigma y = (\frac{\delta f}{\delta x_1} \cdot \sigma x_1)^2 + (\frac{\delta f}{\delta x_2} \cdot \sigma x_2)^2 + \dotsb + (\frac{\delta f}{\delta x_n} \cdot \sigma x_n1)^2 \\
\delta y = \sqrt{\sum\limits_{i=1}^n (\frac{\delta f}{\delta x_i} \cdot \sigma x_i)^2}
\end{multline*}
\\
Wird die Statistische Verteilung der Einzelwerte in einem Histogramm betrachtet, ist die Wahrscheinlichkeit, dass $x$ im im Bereich $x-(x+ \Delta x)$ ist wie folgt: \\
\Large
\begin{align*}
 f(x) = \frac{1}{\sigma \sqrt{2 \Pi}} \cdot e^{\frac{(x-\bar{x})^2}{2 \sigma ^2}}
\end{align*}  
\\
\begin{tabular}{lc}


träge Masse: & \Large $m_t = - \frac{m_{referenz} \cdot v_{referenz}}{v_t}$\\


Kraft & $\overrightarrow{F} = m \cdot \overrightarrow{a}$\\
Gravitation & $\overrightarrow{F_G} = G \cdot \frac{m_1 \cdot m_2 }{r^2}$\\
Elektromagnetische WW: & $\overrightarrow{F_E} = \frac{1}{4 \Pi \cdot \varepsilon_0 \cdot \varepsilon_r} \cdot \frac{Q_1 \cdot Q_2}{r^2}$\\

1. Raketengleichung & $m \cdot \overrightarrow{a} = V_T \cdot \frac{\delta m}{\delta t} + \overrightarrow{g} \cdot m$\\
$V_T$ = v Ausströhm\\ & 
\\
2. Raketengleichung & $V_R(t)-V_R(0) = V_T \cdot ln [\frac{m_t}{m_0}]+\overrightarrow{g} \cdot t$\\
\end{tabular}

\newpage
\textbf{Vorgänge an der festen Rolle}\\

\begin{tabular}{ll}
Zugkraft am Seil: & $Z = g \cdot \frac{2 m_1 \cdot m_2}{m_2+m_1}$\\

Beschläunigung: & $a = \frac{m_2-m_1}{m_2+m_1}$\\
\end{tabular}\\
\\
\textbf{Zugkraft im Kräftediagramm}\\

\includegraphics[width=0.8\textwidth]{/home/joni/Dokumente/Uni/Mittschriften/Physik/Bilder/Kräftediagram}\\

$F_1 = F_2 \cdot \frac{cos(\alpha_2)}{cos(\alpha_1)}$\\
$F_2 = \frac{m \cdot g}{\frac{cos(\alpha_2)}{cos(\alpha_1)} \cdot sin(\alpha_1)+sin(\alpha_2)}$\\


\newpage
Freier Fall\\
$t = \sqrt{\frac{2h}{g}}\\
v = \sqrt{2gh}$\\

bei einem Wurf senkrecht nach oben ist:\\
$H(v_0) = h_0 - \frac{v_0^2}{2g}$\\

 
elastischer Stoß:\\
beide Körper bewegen sich mit unterschiedlicher Geschwindigkeit\\
Sonderfall:\\
beide Massen sind gleich groß: Die Körper tauschen die Geschwindigkeit\\
$v_1' = \frac{m_1 \cdot v_1 + m_2 \cdot (2v_2 - v_1)}{m_1 + m_2}$\\

vollkommen inelastischer Stoß:\\
Beide Körper bewegen sich mit der gleichen Geschwindigkeit (kleben aneinander)\\
$v'= \frac{m_1 \cdot v_1 + m_2 \cdot v_2}{(m_1+m_2)}$\\
$\frac{\Delta E}{E} = \frac{m}{m+M}$\\
\begin{tabular}{ll}
m: & Masse vorher\\
M: & gemeinsame Masse nachher\\
\end{tabular}
\\
Schwerpunkt bei 2 Punktmassen\\
$\overrightarrow{r} = \frac{m_1 \cdot \overrightarrow{r_1} + m_2 \cdot \overrightarrow{r_2}}{m_1 + m_2}$


\newpage
\section{Definitionen}
\begin{enumerate}
\item träge Masse: Fähigkeit eines Körpers sich zu wiedersetzten
\item schwere Masse: Anziehung von Massen durch Gravitation
\item Kraft: ist die Änderung des Impulses
\item Elementare Wechselwirkungen
\begin{center}
\begin{tabular}{l|l|l|l}
Wechselwirkung & Bedeutung & Reichweite & relative Stärke\\
\hline
starke & Kernkraft & kurz & 1\\
elektromagnetische & Coulombkraft & groß & $10^{-2}$\\
schwache & Radioaktivität & kurz & $10^{-13}$\\
Gravitation & Masseanziehung & kurz & $10^{-38}$\\
\end{tabular}
\end{center}
\item Zustandsgröße: eines Körpers bzw. eines Systhems von Massepunkten. \\
\item Mechanische Energie: Beschreibt wie viel Arbeit von dem Körper oder dem Systhem von Massepunktten gelestet werden kann.
\item Kräfte
\begin{itemize}
\item konservative Kräfte: geleistete Arbeit hängt vom Anfangs und Endpunkt der Bewegung ab. (nicht vom (Um)Weg!)\\
Test: $\oint \overrightarrow{F}\delta\overrightarrow{s}=0$
\item nicht-kosnervative Kräfte: Die geleistete Arbeit hängt vom Weg der Bewegung ab.\\
Test: $\oint \overrightarrow{F}\delta\overrightarrow{s}>0$
\end{itemize}
\item innere Energie: z.B Wärme, chemische Energie\\
\end{enumerate}


\newpage
\section{Sätze oder Ähnliches}
Newtonsche Axiome:
\begin{enumerate}
\item Ein körper erfährt keine Beschäunigung, solange keine resultierende äußere Kraft auf ihn wirkt
\begin{itemize}
\item $\overrightarrow{a} = 0$ wenn $\overrightarrow{F} = 0$
\end{itemize}
\item Die Zeitliche Änderung des Impulses ist gleich einer Resultierenden  Kraft die auf ihn wirkt
\begin{itemize}
\item $\frac{\delta P}{\delta t} = m \cdot \overrightarrow{a}$
\end{itemize}
\item Aktion = -Reaktion
\begin{itemize}
\item $F_{AB} = -F_{BA}$
\end{itemize}
\item Energieerhaltungssatz: In einem abgeschlossenen Systhem bleibt die gesammt-Energie erhalten!
\item elastischer Stoß: Die mechanische Energie bleibt erhalten.
\item inelastischer Stoß: Die Gesammtenergie bleibt erhalten
\end{enumerate}


\end{document}