\documentclass[a4paper,12pt]{scrartcl}

\usepackage{amsmath}


\begin{document}
\title{Chemie}
\maketitle

\begin{itemize}

\section{Physikalische Chemie} \label{Physikalische Chemie}

\subsection{Kernspaltung}

\item wird durch Neutronenbeschuss ausgelöst
\item Je größer der Massendefekt, desto stabieler ist der Kern
\item Konkurrenzreaktionen:

\begin{addmargin}[20mm]{0pt}
\begin{itemize}
\item Wenn Neutronen zu schnell eingefangen werden kann kein Kernzerfall stattfinden und das Element wird nur schwerer
\item Die kritische Masse wird unterschritten; bei einer Kettenreaktion werden mehr neutronen durch die Oberfläche verloren als in einen neuen Kern eingebuden
\end{itemize}
\end{addmargin}

\subsection{Kernfusion} \label{Kernfusion}
\item nur unter hohem Druck und Temparatur
\item Problem ist die Handhabung der Temparatur
\end{itemize}

\section{Bindungen}
Metallbindungeny; Koordinationsbindung; Ionenbindung; Kovalentebindung\\
\begin{tabular}{lll}
Leiter & Halbleiter & Isolatop\\
Ag & Si & 
\end{tabular}
\subsection{Koordinationsverbindungen}


\begin{tabular}{ll}
Kristallsystem & orthorhombisch\\
Ausbildung & meist körnig, selten gut ausgebildete Kristallflächen\\
Bruch & muschelig, 1 Spaltbarkeit\\
Glanz & Glasglanz, etwas fettig; transparent\\
Farbe & gelbgrün bis dunkelgrün, Mischkristalle i.d.R. flaschengrün, ockergelb bis bräunlich verwitternd\\
Härte & 7\\
Dichte & 3,3 bis 4,4\\
Vorkommen & in $SiO_2$ armen magmatiten und Metamorphiten; nicht in Paragenese mit Quarz\\
\end{tabular}




\end{document}
